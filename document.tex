\documentclass[11pt]{beamer}
\usepackage[utf8]{inputenc}
\usepackage[T1]{fontenc}
\usepackage{lmodern}
\usepackage{amsmath}
\usepackage{amsfonts}
\usepackage{amssymb}
\usepackage{graphicx}
\usetheme{Boadilla}
\usepackage{media9}
\usepackage{multimedia}
\usepackage{pdfpages}
\usepackage{hyperref}
\usepackage[labelformat=empty]{caption}

\begin{document}
	\author{Julien Nyambal}
	\title{Why GPUs for Machine Learning?}
	%\subtitle{}
	\logo{capitec}
	\institute{Capitec Bank}
	\logo{\includegraphics[height=0.5cm]{capitec.png}}
	%\date{}
	\setbeamertemplate{navigation symbols}{}
	\begin{frame}[plain]
		\maketitle
	\end{frame}

\begin{frame}
	\centering
	While a CPU is the brains of a computer, GPU is its soul ... Nvidia
\end{frame}

	\begin{frame}
		\frametitle{What will be covered ...}
		\tableofcontents
	\end{frame}

\section{What is Machine Learning?}
\begin{frame}
	\frametitle{What is Machine Learning?}
	\begin{figure}
		\includegraphics[width=100mm,scale=0.7]{ml}
		\caption{A general definition of a ML}
	\end{figure}
\end{frame}

\begin{frame}
	\frametitle{What is Machine Learning?: Conceptual Overview}
	\begin{figure}
	\includegraphics[width=100mm,scale=0.7]{ml_concept}
	\caption{ML - Conceptual}
\end{figure}
\end{frame}

\begin{frame}
	\frametitle{What is Machine Learning?: Deep Learning}
	\begin{figure}
		\includegraphics[width=55mm,scale=0.5]{cnn}\hspace{2mm}
		\includegraphics[width=55mm,scale=0.5]{drl}
		\\[\smallskipamount]
		\includegraphics[width=55mm,scale=0.5]{gan}\hspace{2mm}
		\includegraphics[width=55mm,scale=0.5]{rnn}
	\end{figure}
\end{frame}

\section{RAM, CPU, GPU, TPU}
\begin{frame}
	\frametitle{RAM, GPU, CPU, TPU}
	\subsection{Description}
	\begin{center}
		\tiny
	\begin{tabular}{| c | p{2.5cm}| p{2.5cm} | p{2.5cm} | }
		\hline
		\textbf{Acronym} & Central Processing Unit  & Graphical Processing Unit& Tensor Processing Unit \\
		\hline
		\textbf{Description} & Performs basic logic, arithmetic, and I/O operations, and allocate commands to other components and subsystems running in a computer. It is the orchestrator of the other hardware components of the computer system, including the GPU. Most systems have \textbf{a shared GPU resource for basic display}.& \textbf{Dedicated} device designed specifically to accelerate computer graphics workloads, particularly for 3D graphics. While they are still used for their original purpose of accelerating graphics rendering, GPU parallel computing is now used in a wide range of applications, including graphics and video rendering.  & Intel   \\
		\hline
		\textbf{Processing style} & Sequential (But multitask) & Distributed (Given a single heavy-duty load)  & -  \\
		\hline
		\textbf{Graphic Rendering API} & OpenCL& CUDA  & -  \\
		\hline
		\textbf{Specialty} & Orchestrating several tasks performed by the OS &	\begin{itemize}
			\item Graphically intense video game Image Rendering
			\item Distributed Complex matrix operation: \textbf{Machine Learning} and \textbf{HPC}
			\item Video Editing 
		\end{itemize} & -  \\
		\hline  
	\end{tabular}
\end{center}
\end{frame}

\section{RAM, CPU, GPU, TPU}
\begin{frame}
	\frametitle{RAM, GPU, CPU, TPU}
	\subsection{TPU: Tensor Processing Unit}
	\begin{figure}
		\includegraphics[width=50mm,scale=0.5]{ram5}\hspace{2mm}
		\includegraphics[width=50mm,scale=0.5]{v100}
		\\[\smallskipamount]
		\includegraphics[width=50mm,scale=0.5]{cpu2}\hspace{2mm}
		\includegraphics[width=50mm,scale=0.5]{tpu}
		\caption{Some images}\label{fig:foobar}
	\end{figure}
\end{frame}

\section{Hardware Comparison CPU vs GPU}
\begin{frame}
	\frametitle{Hardware Comparison CPU vs GPU}
	\subsection{Hardware Comparison CPU vs GPU}
	\begin{figure}
		\includegraphics[width=\textwidth,height=\textheight,keepaspectratio]{cpu_vs_gpu}
	\end{figure}
\end{frame}

\begin{frame}
	\frametitle{Hardware Comparison CPU vs GPU}
	\begin{center}
		\begin{tabular}{| c | c | c |}
			\hline
			\textbf{Processor Type} & \textbf{CPU} & \textbf{GPU} \\ 
			\hline
			\textbf{Processor Model} & i9-10900K & Tesla V100 \\  
			\textbf{Manufacturer} & Intel & Nvidia  \\
			\textbf{Processor Speed (Max)} & 5.30 GHz & 1.380 GHz  \\
			\textbf{Memory (up to)} & 128 GB (RAM) & 32 GB (GRAM) \\
			\textbf{Number of Cores} & 10 & 5120 (CUDA cores) \\
			\textbf{Memory Bandwidth} & 45.8 GB/s & 900 GB/s\\
			\textbf{Price} & \textasciitilde R 16 000 & \textasciitilde R 200 000\\
			\hline  
		\end{tabular}
	\end{center}
\end{frame}

\begin{frame}
	\frametitle{Tensors: Operations - Addition}
	%\subsection{Addition}
		 \begin{figure}
		\includegraphics[scale=0.17]{"1 - original_image"}
		\caption{Original Image}
	\end{figure}
\end{frame}

\begin{frame}
	\frametitle{Tensors: Operations - Addition}
	%\subsection{Addition}
	\begin{figure}
		\includegraphics[scale=0.17]{"2 - gauss_noise"}
		\caption{Noise}
	\end{figure}
\end{frame}

\begin{frame}
	\frametitle{Tensors: Operations - Addition}
	%\subsection{Addition}
	\begin{figure}
		\includegraphics[scale=0.17]{"3 - gn_img"}
		\caption{Image + Noise}
	\end{figure}
\end{frame}

\begin{frame}
	\frametitle{Tensors: Operations - Addition}
	%\subsection{Addition}
	\begin{figure}
		\includegraphics[scale=0.17]{"4 - gn_img_avg"}
		\caption{(Image + Noise) + Image}
	\end{figure}
\end{frame}

\begin{frame}
	\frametitle{Tensors: Operations - Matrix Multiplication}
	%\subsection{Matrix Multiplication}
\begin{itemize}
	\item The most used operation in Machine Learning/Deep Learning
	\item There are many types of Matrix Multiplication of \textit{\textbf{MatMul}}. There 3 are the most common matrix multiplication:
	\begin{itemize}
		\item Brute Force Multiplication
		\item Column-Wise Multiplication
		\item Block Multiplication		
	\end{itemize}
\end{itemize}
\end{frame}

\begin{frame}
	\frametitle{Tensors: Operations - Matrix Multiplication}
	%\subsection{Matrix Multiplication}
	\begin{figure}
		\includegraphics[scale=0.17]{"1 - original_image"}
		\caption{Image}
	\end{figure}
\end{frame}

\begin{frame}
	\frametitle{Tensors: Operations - Matrix Multiplication}
	%\subsection{Matrix Multiplication}
	\begin{figure}
		\includegraphics[scale=0.17]{"conv_1"}
		\caption{Convolution 1}
	\end{figure}
\end{frame}

\begin{frame}
	\frametitle{Tensors: Operations - Matrix Multiplication}
	%\subsection{Matrix Multiplication}
	\begin{figure}
		\includegraphics[scale=0.17]{"conv_2"}
		\caption{Convolution 2}
	\end{figure}
\end{frame}

\begin{frame}
	\frametitle{Tensors: Operations - Matrix Multiplication}
	
	\begin{figure}
		\includegraphics[width=100mm,scale=0.5]{brute_force_mat_mul}
		\\[\smallskipamount]
		\includegraphics[width=100mm,scale=0.5]{column_wise_mat_mul}
		\\[\smallskipamount]
		\includegraphics[width=100mm,scale=0.5]{block_multiplication}
	\end{figure}
	
%			\begin{figure}
%				\includegraphics[width=50mm,scale=0.5]{brute_force_mat_mul}
%			\end{figure}
%
%			\begin{figure}
%				\includegraphics[width=50mm,scale=0.5]{column_wise_mat_mul}
%			\end{figure}
%
%			\begin{figure}
%				\includegraphics[width=40mm,scale=0.5]{block_multiplication}
%			\end{figure}
\end{frame}

\begin{frame}
	\frametitle{Tensors: Operations - Matrix Multiplication}
	\centering
\includegraphics[width=\textwidth+2cm,height=\textheight+2cm,keepaspectratio, angle = -90 ]{demo.pdf}
\end{frame}

\section{Tensor Operations on Hardware}
\begin{frame}{movie}
	\frametitle{Matrix Computation on GPU vs CPU}
	\subsection{Matrix Computation on GPU vs CPU}
	\begin{figure}[h!]
	\centering
	{\includegraphics[width=1.0\textwidth]{gpugif.png}}
	\end{figure}
\end{frame}

\section{How does GPU works "faster" than the CPU?}
\begin{frame}
	\frametitle{How does GPU works "faster" than the CPU?}
	\begin{itemize}
		\item \textbf{Larger memory bandwidth}: More data can get in and out the component at a time,
		\item \textbf{Parallelization}: Those many CUDA cores work well in parallel for one given task
		\item \textbf{Fast Memory access}: Multiple L1 and L2 Cache memory,
		\item CUDA API allows you not to worry about memory allocation or deallocation, type of Matrix Multplication to use or how to orchestrate the parallelism of the cores. Some popular frameworks using CUDA:
			 \begin{figure}
			\includegraphics[width=\textwidth,height=\textheight,keepaspectratio]{dl_frameworks}
		\end{figure}
	\end{itemize}
\end{frame}

\section{Very short demo}
\begin{frame}[fragile]
	\frametitle{Demo}
	
	We will run short 2 experiments on both the CPU and the GPU:
	\begin{itemize}
		\item Multiplication of 2 scalar: 8.8 x 8.9
		\item Multiplication of 2 relatively big matrices
	\end{itemize}
	
		\centering
	Demo Link: 
\href{https://colab.research.google.com/drive/1q32stN5vWQ3XApmu58BczJJy62wTI3Zf#scrollTo=UjuLyp2wMK7D}{Colab Experiment}.
\end{frame}

\end{document}